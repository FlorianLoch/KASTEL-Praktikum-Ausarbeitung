\section{Feldstudie}
Um die Praktikabilität der Angriffe zu verifizieren haben wir ihre Erfordernisse (z.B. das Aussenden von Probe-Requests) in mehreren Feldstudien überprüft.
Hierfür wurde der Netzwerkverkehr in variierenden Umgebungen, sowie das Verhalten von Endgeräten gegenüber den Angriffen analysiert.
Die Leitfragen des folgenden Kapitels sind dabei:
\begin{itemize}
	\item Welche Endgeräte senden in welcher Frequenz Probe-Requests?
	\item Wie anfällig sind Endgeräte bezüglich dem Evil-Twin Angriff?
	\item Können Deauthentication-Angriffe im eigenen Netzwerk erkannt werden?
\end{itemize}

\subsection{Probe-Request Verhalten der Endgeräte}
Um einen Evil-Twin Angriff zu ermöglichen benötigt der Angreifer eine oder mehrere SSIDs von Netzwerken, die dem anzugreifenden Endgerät bekannt sind.
Hierfür wurden die Probe-Requests zu Hilfe gezogen, die Endgeräte periodisch aussenden um bekannte Netzwerke zu detektieren.
Diese variieren jedoch in ihrer Frequenz und Fülle abhängig vom Endgerät.
Wir haben Smartphones und Laptops verschiedener Hersteller auf ihr Verhalten geprüft:
\begin{itemize}
	\item Nokia Lumia 530 (Windows 8.1)
	\item Elephone P9000, Samsung Galaxy S7, Samsung Galaxy S6 Edge (Android 6)
	\item Lenovo Thinkpad T550 (Windows 10)
	\item Apple MacBook Pro 15 (Mid 2015), Apple MacBook Pro 13 () %TODO!
\end{itemize}
Der Netzwerkverkehr wurde mithilfe von \texttt{airodump-ng} aufgezeichnet und nachträglich mit eigens entwickelten Python-Scripten auf Grundlage von \texttt{Pyshark} analysiert und ausgewertet.


Windows 8.1 sowie Android 6 Smartphones wiesen einen ähnlichen Probe-Request-Verlauf auf und sendeten bis zum 40 Anfragen pro Minute aus.
Auffällig für die getesteten Android-Geräte war jedoch, dass mit steigender Anzahl bekannter SSIDs die Anzahl der Probe-Requests stiegen.
Beim überwachen des relevanten Netzwerkverkehrs an öffentlichen Plätzen wurde die Auskunftsfreudigkeit von Android-Endgeräten bestätigt\footnote{Diese Werte sind natürlich nur approximativ, da wir aufgrund des Herstellers (Vendor-MAC) auf das Endgerät schließen mussten und Nutzerpräferenzen nicht berücksichtigen konnten.}. %TODO: noch was wegen MAC-Randomization erwähnen? Machts halt auch schwerer
Weiterhin tendieren Android-Geräte dazu alle bekannten SSIDs burstartig binnen weniger hundert Millisekunden zu versenden.
Das Elephone P9000 sendete sogar weiter Probe-Requests, obwohl es mit einem Wireless-Netzwerk verbunden war.
Wir vermuten an dieser Stelle, dass die Ursache Location-Services sind oder das Ziel, stärkere Netzwerke in Reichweite zu finden.
Die Apple-Geräte sowie der Windows 10 Laptop hingegen schienen sparsamer mit Probe-Requests umzugehen.
Diese sendeten lediglich drei bis zehn Anfragen pro Minute aus.
Abschließend ist zu sagen, dass keines der getesteten Geräte gänzlich auf Probe-Requests verzichtet hat.
Daher ist die Vorbereitung des Evil-Twin Angriffs als durchaus praktikabel zu bewerten.

\subsection{Anfalligkeit der Endgeräte: Evil Twin}
Nachdem festgestellt wurde, dass wir mit wenig Aufwand SSIDs aus Probe-Requests des Endgeräts erhalten können, wollen wir nun prüfen, ob eventuelle Schutzmechanismen gegen einen Evil-Twin Angriff existieren.
Wir verwenden die Geräte des vorherigen Tests um ein möglichst breites Herstellersprektrum abzudecken.
Auf allen Endgeräten wurde das WPA2-PSK geschützte Netzwerk mit der SSID \enquote{Tor-zum-Internet} zu der Liste bekannter Netzwerke hinzugefügt.
Anschließend wurde, wie in der Durchführung erläutert, ein passender Evil-Twin mit der selben SSID via \texttt{airbase-ng} mit der passenden Konfiguration gestartet.
Die Endgeräte wurden während der Durchführung des Angriffs nicht bedient und waren nicht in Reichweite anderer bekannter Netzwerke.

An dieser Stelle ist es wichtig nochmals zu erwähnen, dass der Handshake im Schritt 3 vom Evil-Twin abgelehnt wird.
Daraus resultiert unter anderem das unterschiedliche Verhalten der einzelnen Android-Endgeräte trotz gleicher Betriebssystemversion.
So zeichnete sich das Samsung Galaxy S6 Edge durch ein aggressives Authentifizierungsverhalten aus: Mit einer Rate von ca. zwei Authentifizierungen pro Sekunde stoch dieses Endgerät besonders heraus.
Da es sich immer mit dem stärksten verfügbaren Netzwerk verbunden hat, hatte das Smartphone keinen Internetzugang mehr (z.B. über mobiles Datenvolumen).
In diesem Fall konnte der Evil-Twin also für einen Denial-of-Service Angriff missbraucht werden.
Die übrigen Endgeräte haben nach zwei bis vier gescheiterten automatischen Authentifizierungen aufgegeben und sind auf die Nutzung von mobilem Internet zurückgefallen.
Dieses Verhalten trat jedoch periodisch in Zeitabständen von 30 bis 60 Sekunden auf.
Die einzige Ausnahme bildeten hier die Laptops die ohne Nutzeraktion keine weiteren Handshakes initiiert haben.

Sind bekannte Alternativnetzwerke in Reichweite, so bevorzugen die meisten Endgeräte das stärkste Netzwerk und wechseln wenn nötig den Access-Point.
Wir konnten dieses Verhalten bei allen Endgeräten Ausnutzen, um Endgeräte zur Authentifikation mit dem Evil-Twin zu verleiten.
Nach dem gescheiterten Handshake sind sie jedoch wieder zum Alternativnetzwerk gewechselt.

Ist der Typ des Netzwerks vorher nicht bekannt kann der Angreifer, wie in der Durchführung erwähnt, mehrere Evil-Twins auf dem Netzwerkinterface öffnen.
Alle getesteten Endgeräte haben sich in diesem Fall mit dem Access-Point mit der passenden Sicherheitskonfiguration verbunden.
Wurde absichtlich die falsche Konfiguration gewählt, so haben dies lediglich das Windows 8.1 Phone und der Windows 10 Laptop erkannt und den Nutzer entsprechend gewarnt. %TODO Bild hier
Die übrigen Endgeräte haben sich somit in einem offenen Evil-Twin ohne jegliche Verschlüsselung locken lassen, womit einem Man-in-the-Middle Angriff Tür und Tor geöffnet wird.

Zusammenfassend ist der Evil-Twin als ein wenig subtiler, jedoch hochflexibler Angriff zu bewerten.
Jedes der geprüften Endgeräte sendet die notwendigen SSIDs aus was den Angriff auf ein konkretes Netzwerk aber auch den Angriff auf einen konkreten Client leicht gestaltet.
Weiterhin ist der Evil-Twin der einzige Angriff der Off-site eingesetzt werden kann. 
Somit ist kein realer Access-Point für das Erhalten des teilweisen Handshakes nötig.
Eigenen Tests nach harmoniert der Evil-Twin Angriff sehr gut mit dem Deauthentication-Angriff.
So kann ein bereits mit einem anderen Netzwerk verbundener Client so lange deauthentifiziert werden, bis er sich mit dem Evil-Twin verbindet.
Der einzige Nachteil bleibt jedoch, dass der Evil-Twin im Netzwerk für alle sichtbar ist und somit leicht erkannt werden kann.

%(geht ein wenig einher mit dem Aussenden bekannter SSIDs, aber nicht unbedingt, da ich den Evil Twin Angriff auch auf Stationen anwenden kann, die gerade mit einem AP verbunden sind)
%Nicht alle Endgeräte connecten automatisch mit dem Evil Twin
%Windows Phone scheint sich nicht nur die SSID+MAC (die wir beide spoofen können) zu merken, sondern auch gewisse Netzwerkeinstellungen für den AP ("airbase-ng -Z 2" flag setzten um WPA2 AP aufzubauen)

%Des weiteren verwenden Apple-Geräte MAC-Randomization um ihre eindeutige Identität zu verschleiern und gezielte Angriffe gegen sie zu erschweren.
%Dieser Mechanismus hat jedoch keine Auswirkung a

\subsection{Deauthentication-Verhalten von APs}
Da keines der getesteten Endgeräte Schutzmechanismen gegenüber dem Deauthentication-Angriff aufgezeigt hat verzichten wir auf eine ausführliche Beschreibung und fokussieren uns auf die Erkennung des Angriffs und auf seine Subtilität.
Dafür wurde in dieser Feldstudie der Netzwerkverkehr in zwei verschiedenen Szenarien aufgezeichnet:
\begin{enumerate}
	\item \textbf{WPA2-PSK Heimnetzwerk} mit einem Access-Point und 8 Teilnehmern, die sich selten aus der Reichweite des Access-Points entfernen
	\item \textbf{WPA2-Enterprise Firmennetzwerk} mit 3 Access-Points und ca. 30 Teilnehmern, die sich häufig zwischen den einzelnen Access-Points bewegen
\end{enumerate}
Wir gehen davon aus, dass das Management des Firmennetzwerks, also unter anderem das Aussenden von Deauthentication-Frames, unabhängig vom konkreten WPA2-Authentifizierungsverfahren ist.
Daher ist anzunehmen, dass sich ein WPA2-Enterprise geschütztes Netzwerk ähnlich verhält wie ein WPA2-PSK geschütztes.
Aus den Merkmalen des Verkehrs, wie beispielsweise Typ, Häufigkeit und Inhalt der Deauthentication-Frames, können wir Eigenschaften ableiten mit denen wir den Deauthentication-Angriff bezüglich seiner Praxistauglichkeit bewerten können.

Im Heimnetzwerk konnte selbst nach 24 Stunden des Aufzeichnens kein Deauthentication-Frame beobachtet werden.
Unsere Erklärung dafür ist, dass in einem Heimnetzwerk mit einem einzigen Access-Point keine Notwendigkeit für das Umverteilen von Clients besteht.
Weiterhin erfolgen Neuauthentifizierung durch Neustart der Endgeräte bzw. durch ausreichende Entfernung vom Access-Point und muss daher nicht forciert werden.
Folglich ist ein Deauthentication-Angriff hier auch mit wenigen gesendet Frames äußerst auffällig und mit entsprechender Hardware leicht zu erkennen.

Im Firmennetzwerk hingegen wurden im Schnitt 2,2 Deauthentication-Frames pro Minute detektiert.
Hierbei sind ausschließlich folgende Reason-Codes aufgetreten:
\begin{itemize}
	\item 1: Unspecified reason $\rightarrow$ kein genauer Grund angegeben
	\item 2: Previous authentication no longer valid $\rightarrow$ der Schlüssel des verbundenen Clients soll zugunsten der Vertraulichkeit neu ausgehandelt werden
	\item 6: Class 2 frame received from nonauthenticated STA $\rightarrow$ der Client hat versucht vor der Authentifizierung Daten zu übertragen
	\item 8: Disassociated because sending STA is leaving (or has left) BSS $\rightarrow$ das Management-System hat entschieden den Client einer anderen Station zuzuordnen (Load-Balancing)
\end{itemize}
Auffällig war ebenfalls, dass die Access-Points bis zu 5 Deauthentication-Frames gleichzeitig in einem Burst gezielt an einen Client versendet haben.
Dieses Verhalten kann dem Angreifer dabei helfen seinen Angriff zu verschleiern.
So kann er zum Beispiel den Netzwerkverkehr für kurze Zeit analysieren um zu erkennen, mit welcher Frequenz und mit welchen Reason-Codes die Frames von den Access-Points im Zielnetzwerk versendet werden.
In einem Firmennetzwerk ist somit der Deauthentication-Angriff als praktikabel und auch subtil zu bewerten wenn der Angreifer die nötige Zeit aufwendet um seinen Angriff entsprechend den Gegebenheiten anzupassen.

Der Deauthentication-Angriff ist für den Client selbst kaum nachvollziehbar: es kommt lediglich zu einem kurzen Aussetzer in der Verbindung zum Access-Point, da die Neuauthentifizierung binnen weniger 100 Millisekunden abläuft.
Geht man davon aus, dass der Nutzer das Smartphone im Moment der Deauthentifikation benutzt, so ist der entstehende Verbindungsabbruch für ihn in der Statusleiste erkennbar.
Der Angreifer kann dies jedoch vermeiden indem er auf ein Zeitfenster wartet, in dem keine Datenpakete zwischen dem Access-Point und dem Client ausgetauscht werden.
Hier kann er davon ausgehen, dass der Client das Endgerät momentan nicht bedient und eventuell gesperrt hat.

Abschließend ist zu sagen, dass selbst die Erkennung eines Deauthentication-Angriffs nicht davor schützt, dass der Angreifer die Handshakes abfängt und bricht.
Der Netzwerkadministrator müsste nach einem offensichtlichen Angriff den PSK neu festlegen und an alle Clients verteilen.
Da diese Prozedur zu aufwändig ist, sollte durch ein ausreichend sicheres Passwort der Prozess des Handhshake-Bruteforcings nahezu unmöglich gemacht werden.


%Sollte nur relativ kurz werden, also welche Szenarien haben wir betrachtet (Heimnetzwerke, Bahnhof mit öffentlichen Netzwerken, AP in Firmennetzwerk) und wie oft/üblich sind deauths (im Heimnetzwerken von Haus mit 7 APs kein einziges deauth Paket in c.a. 24h, Firmennetzwerk mit c.a. 30 Teilnehmern in Reichweite des Accesspoints: rund 2,2 deauth Pakete pro Minute, reason codes variieren)
