\section{Motivation \& Einleitung}
Drahtlose Netzwerke, insbesondere IEEE 802.11, sind seit Jahren allgegenwärtig und werden in ihrer Bedeutung sicher noch weiter zunehmen. Egal ob in Büroräumen, in Universitäten oder dem eigenen Wohnzimmer - WLAN, ist nicht wegzudenken -- so verfügen viele Endgeräte mittlerweile auch gar nicht mehr direkt über tatsächliche Anschlüsse für kabelgebundene Netzwerke; ganze andere Klassen von Endgeräten wurden erst durch WiFis möglich bzw. nützlich: Smartphones, Tablets etc..

Wohl niemand hat ernsthafte Bedenken, wenn er in drahtlosen Netzen kommuniziert -- ungeschützte, öffentliche Hotspots einmal ausgelassen. Und dass, obwohl gerade diese Netze aus physikalischer Sicht besonders exponiert sind -- die schlichte Nähe zum Zugangspunkt genügt, um aktiv in die Kommunikation einzugreifen.

Doch durch entsprechenden Schutz auf Schicht zwei des ISO/OSI-Modells sollen Vertraulichkeit und Integrität aller darüberliegenden Protokolle transparent gewahrt werden.

Nachdem mit WEP ein nur unzureichendes Verfahren entwickelt und auf den Markt gebracht worden war, um diese Ziele umzusetzen, wurde mit WPA eine Teilmenge der damals in Entwicklung befindlichen \enquote{Robust Secure Network}-Spezifikation (RSN bzw. 802.11i TODO Quelle) als vorübergehender Standard verabschiedet. Seit 2004 ist der komplette RSN-Standard unter dem Namen WPA2 dabei, die alten Verfahren vollständig zu ersetzen.	

Bezüglich der grundlegenden, theoretischen Sicherheit von WPA2(-PSK) und den kryptographischen Primitiven gibt es in der Wissenschaft keinen Zweifel \footnote{2012 wurde das Verfahren nach 8 Jahren Einsatz weiterhin als sicher angesehen \cite{kumkar2012}, seitdem sind keine nennenswerten Schwachstellen gefunden worden.} -- diese Arbeit widmet sich daher der Frage, wie praxisrelevant Bruteforce-Angriffe auf WPA2-PSK sind, wie sie funktionieren und ob daher in der Praxis tatsächlich weiterhin sorglos auf 802.11i gesetzt werden sollte. Auf WPA2-EAP, welches statt eines \textit{Public Shared Keys} (PSK) das \textit{Extended-Authentification-Protocol} (EAP) zur Authentifizierung verwendet, wird nicht weiter eingegangen werden.

Hierfür werden nötige Grundlagen hinsichtlich IEEE 802.11 eingeführt, anschließend werden verschiedene Angriffsansätze zur Erlangung eines WPA2-PSK-Handshakes sowie das (versuchte) Brechen desselben beschrieben. Dabei wird neben der konzeptionellen Herangehensweise auch die praktische Durchführung mit konkreten Programmen aufgezeigt. Wir schließen mit einer Zusammenfassung unserer Ergebnisse und einer Diskussion hinsichtlich der Praktikabilität der vorgestellten Angriffe.
