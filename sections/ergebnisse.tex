\section{Ergebnisse}
Hier kommt der zweite (und der womöglich größte) Teil der Arbeit rein, nämlich die Ergebnisse unserer Studien:

\subsection{Welche Endgeräte senden SSIDs bekannter APs?}
Welche Endgeräte senden alle bekannten APs mit? (Einteilung nach Hersteller möglich)  
Ist die Frequenz der Probes interessant?

\subsection{Anfalligkeit der Endgeräte: Evil Twin}
(geht ein wenig einher mit dem Aussenden bekannter SSIDs, aber nicht unbedingt, da ich den Evil Twin Angriff auch auf Stationen anwenden kann, die gerade mit einem AP verbunden sind)
Nicht alle Endgeräte connecten automatisch mit dem Evil Twin
Windows Phone scheint sich nicht nur die SSID+MAC (die wir beide spoofen können) zu merken, sondern auch gewisse Netzwerkeinstellungen für den AP ("airbase-ng -Z 2" flag setzten um WPA2 AP aufzubauen)

\subsection{Deauthentication-Verhalten von APs}
Sollte nur relativ kurz werden, also welche Szenarien haben wir betrachtet (Heimnetzwerke, Bahnhof mit öffentlichen Netzwerken, AP in Firmennetzwerk) und wie oft/üblich sind deauths (im Heimnetzwerken von Haus mit 7 APs kein einziges deauth Paket in c.a. 24h, Firmennetzwerk mit c.a. 30 Teilnehmern in Reichweite des Accesspoints: rund 2,2 deauth Pakete pro Minute, reason codes variieren)
Welche Reason Codes sind üblich, was ist die jeweilige Erklärung für das auftreten eines konkreten Reason Codes?
\begin{itemize}
	\item 1: unspecified reason 
    \item 2: previous authentication no longer valid --> Client has associated but is not authorised
    \item 6: Class 2 frame received from nonauthenticated STA --> Client attempted to transfer data before it was authenticated
   	\item 8: Disassociated because sending STA is leaving (or has left) BSS --> Operating System moved the client to another access point using non-aggressive load balancing (OS hat entschieden den AP zu wechseln)  
\end{itemize}