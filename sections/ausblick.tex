\section{Zusammenfassung \& Ausblick}
In dieser Ausarbeitung wurde gezeigt, wie ein Angreifer mit geringem Hardwareaufwand effizient WPA2-PSK Handshakes eines Zielnetzwerks oder Zielclients erhalten kann. 
Diese können im Anschluss dazu verwendet werden, um den Netzwerkschlüssel offline zu erraten.

Da bislang keine Möglichkeit zur Unterscheidung eines gefälschten Deauthentication-Frames von einem validen besteht und auch sonst keine Schutzmechanismen gegen den Deauthentication-Angriff existieren bedarf es der Erweiterung des Standards 802.11 um die Authentifizierung von Management-Frames. 
Eine solche Erweiterung existiert in Form des Standards 802.11w~\cite{Ahmad:2011:SPS:1998412.1998424}, welcher bisher jedoch nur auf wenigen Endgeräten und Access-Points Unterstützung findet.\\

Mit Hilfe des Evil-Twin Angriffs kann eine Angreifer ohne einen AP in Reichweite den Client durch Simulation eines solchen zu einem Handshake verleiten.
Ein daraus folgender Vorteil dieses Verfahrens ist, dass gezielt für einen Client mehrere Handshakes für alle bekannten Netzwerke gesammelten werden können, solange er diese via Probe-Requests bekannt gibt.
Um nun beispielsweise einen Man-in-the-Middle Angriff durchzuführen muss der Angreifer nicht etwa einen großen, komplexen Passwortraum explorieren, sondern lediglich das am schwächsten geschützte, dem Client bekannte Netzwerk finden. Die praktische Erprobung und zuverlässige Automatisierung dieses Ansatzes erscheint durchaus vielversprechend.
Ein besserer Umgang mit dem Probe-Request-Response-Mechanismus könnte ihr sehr viel zusätzliche Sicherheit bringen, auch aus Sicht des Datenschutzes.\\

Beide vorgestellten Angriffe bauen auf der Annahme auf, dass Clients mit schwach geschützten Netzwerken verbunden sind und ein mitgeschnittener WPA2-PSK Handshake daher in realistischer Zeit zu brechen ist. Die Sicherheit von 802.11-Netzen grundlegend sicherstellen könnte die Verwendung eines sicheren, zertifikatbasierten Handshake-Verfahren, bei dem sowohl der Client als auch der AP authentifiziert werden.
Der nötige Einrichtungsaufwand und Kostenfaktor, u. A. durch den Bedarf der Signierung der Zertifikate, spricht jedoch oft gegen die Nutzung eines solchen Authentifizierungsmechanismus.
Letztendlich liegt die Verantwortung zur Sicherung des WPA2-PSK Netzwerks in den Händen des Administrators, denn die Verwendung eines starken Passworts bleibt der einzig wirksame Schutz gegen Handshake-Bruteforcing bei WPA2-PSK. 
