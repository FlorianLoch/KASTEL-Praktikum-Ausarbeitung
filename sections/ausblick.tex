\section{Ausblick}
In dieser Ausarbeitung wurde gezeigt, wie ein Angreifer mit geringem Hardwareaufwand effizient WPA2-PSK Handshakes zwischen Client und Access-Point erhalten kann, um diese nachträglich mit geeigneten Tools zu brechen.
Befindet sich der Angreifer vor Ort, so kann der Deauthentication-Angriff genutzt werden um einen erneuten Handshake zwischen den beiden Teilnehmern zu forcieren.
In verwalteten Netzwerken, in denen Deauthentication-Frames zum Einsatz kommen, kann dieser Angriff durchgeführt werden, ohne dass der Client die Möglichkeit besitzt eine valide Deauthentifizierung von einer gefälschten zu unterscheiden.
Ist der Handshake gebrochen, so hat der Angreifer mit dem erratenen PSK Zugriff auf das interne Netzwerk. 
Des weiteren lässt sich der komplette Nachrichtenverlauf, der seit dem Handshake aufgezeichnet wurde, nachträglich entschlüsseln.
Dies gilt auch für zukünftigen Netzwerkverkehr anderer Clients unter der Voraussetzung, dass auch hier der Handshake mitgeschnitten wurde.
Da der PSK bekannt und die übrigen Parameter öffentlich sind kann der PTK direkt berechnet werden.
Da keine Schutzmechanismen gegen den Deauthentication-Angriff existieren bedarf es der Erweiterung des Standards um den Deauthentication-Frame zu authentifizieren. %TODO Welcher für WPA2?

Alternativ kann der Angreifer mithilfe des Evil-Twin Angriffs ohne einen echten Access-Point in Reichweite den Client zum Handshake verleiten.
Ein weiterer Vorteil dieses Verfahrens ist, dass gezielt für einen Client mehrere Handshakes für alle bekannten Netzwerke gesammelten werden können, solange er diese via Probe-Requests bekannt gibt.
Um nun beispielsweise einen Man-in-the-Middle Angriff durchzuführen muss der Angreifer nicht etwa einen großen, komplexen Passwortraum explorieren, sondern lediglich das am schwächsten geschützte, dem Client bekannte Netzwerk finden.
Ein Client kann sich schützen, indem er auf dem Endgerät die MAC-Adressen bekannter Access-Points speichert denen er vertraut. 
Damit kann zumindest einem breitflächigen Off-Site Angriff entgegengewirkt werden, denn ist der Angreifer in Reichweite der vertrauten Access-Points kann er deren MAC-Adressen aus verschiedensten Frames extrahieren.
In einem Netzwerk mit vielen Access-Points ist die Pflege dieser Einträge jedoch mühselig und zu aufwändig.
%TODO:Noch irgendwas von wegen Probe Requests sind unnötig

Abschließend ist zu erkennen, dass beide vorgestellten Angriffe auf der Annahme aufbauen, dass Clients mit schwach geschützten Netzwerken verbunden sind und ein mitgeschnittener WPA2-PSK Handshake daher leicht zu brechen ist.
Im Umkehrschluss ist der einzig wirksame Schutz die Verwendung eines starken Passworts oder eines sichereren Handshakes, der sowohl den Client als auch den Access-Point authentifiziert.


%Hinweis, dass über Replay-Attacken mit getrennten Funkstrecken auch ohne Brechen des eigentlichen Handshakes eine Authentifizierung erfolgen kann, diese jedoch nichts nutzt, da die weitere Kommunikation mit PSK erfolgt (Ist das so? Wird kein neuer Schlüssel ausgehandelt?).
%Was kann mit Schlüssel angestellt werden? Wieso sollte man ihn haben wollen? 
%Wie kann ich den Payload einer Nachricht als Evil Twin mitlesen? 